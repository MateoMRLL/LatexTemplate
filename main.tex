%% main.tex
%% Document d'exemple général pour montrer toutes les fonctionnalités
%% Compile avec: pdflatex -shell-escape main.tex

\documentclass[11pt]{rapport}

\usepackage{lipsum}
\usepackage{url} % Ajout pour \url

\title{Exemple de Rapport Complet}

% Si vous utilisez biblatex, décommentez : 
%\addbibresource{./biblio.bib} 
%\makeglossaries 
%\loadglsentries{glossary}

\begin{document}

%% ========================================
%% INFORMATIONS DU RAPPORT
%% ========================================
\logo{./illustrations/logo.png}
\unif{Université Exemple}
\titre{Exemple de Rapport Complet}
\cours{Cours Exemple}
\sujet{Sujet Exemple}
\enseignant{Prof. Exemple}
\eleves{Élève 1 \\ Élève 2} % Correction de l'accent

\fairemarges
\fairepagedegarde

%% ========================================
%% TABLE DES MATIÈRES, FIGURES, TABLEAUX
%% ========================================
\clearemptydoublepage
\renewcommand{\contentsname}{Table des matières}
\tableofcontents
\newpage
\listoffigures
\newpage
\listoftables

%% ========================================
%% INTRODUCTION
%% ========================================
\clearemptydoublepage
\section{Introduction}

Ce document présente un exemple générique illustrant toutes les fonctionnalités de la classe \texttt{rapport}.

\subsection{Objectifs}
Les objectifs principaux sont :

\begin{itemize}
    \item Illustrer les sections, sous-sections et sous-sous-sections
    \item Montrer des listes à puces et numérotées
    \item Inclure des tableaux et figures
    \item Illustrer les environnements de code et blocs colorés
    \item Gérer les annexes et bibliographie
\end{itemize}

\begin{alertblock}{Remarque importante}
Cet exemple est complètement générique et prêt à être adapté.
\end{alertblock}

%% ========================================
%% CONTENU TECHNIQUE
%% ========================================
\section{Contenu Technique}

\subsection{Exemple de figure}
% Vérifiez que le fichier existe ou commentez cette ligne
% \insererfigure{./illustrations/img_example.png}{5cm}{Exemple de figure}{fig_exemple}

\subsection{Exemple de tableau}

\begin{table}[ht]
\centering
\caption{Tableau d'exemple}
\label{tab_exemple}
\begin{tabular}{@{}lcr@{}}
\toprule
Nom & Quantité & Prix \\
\midrule
Produit A & 10 & 5,00 \\
Produit B & 20 & 3,50 \\
Produit C & 5 & 12,00 \\
\bottomrule
\end{tabular}
\end{table}

\subsection{Environnements de code}

\begin{code}{Exemple de code C}{c}
#include <stdio.h>

int main() {
    printf("Hello, world!\n");
    return 0;
}
\end{code}

\begin{codecourt}{python}
def hello():
    print("Bonjour, Python !")
\end{codecourt}

\subsection{Blocs colorés}

\begin{exampleblock}{Exemple de note}
Ceci est un bloc informatif.
\end{exampleblock}

\begin{alertblock}{Alerte !}
Ceci est un bloc d'alerte.
\end{alertblock}

\begin{block}{Information générale}
Ceci est un bloc d'information standard.
\end{block}

%% ========================================
%% MÉTHODES ET ANALYSES
%% ========================================
\section{Méthodes et Analyses}

\subsection{Liste numérotée}

\begin{enumerate}
    \item Étape 1 : Description
    \item Étape 2 : Description
    \item Étape 3 : Description
\end{enumerate}

\subsection{Équations}

Exemple d'équation inline : \(E=mc^2\).  
Exemple d'équation centrée :
\[
f(x) = \int_0^\infty e^{-t} t^{x-1} dt
\]

\subsection{Notes et séparateurs}
\sepline

Du texte supplémentaire pour montrer les séparateurs.

\sepstars

Encore du texte pour illustrer les séparateurs étoilés.

%% ========================================
%% ANNEXES
%% ========================================
\clearemptydoublepage

\appendix

\annexe{Code complet}
\begin{code}{Fichier principal main.c}{c}
#include <stdio.h>

int main(void) {
    printf("Exemple de code complet en annexe.\n");
    return 0;
}
\end{code}

\annexe{Exemple supplémentaire}
\lipsum[1-3]

\annexe{Languages}
\selectlanguage{french}
Ceci est un texte en français.

\selectlanguage{english}
% Commenté car la référence dummy n'existe pas
This is the same text in English. % \cite{dummy}

\listofannexes

%% ========================================
%% BIBLIOGRAPHIE
%% ========================================

\clearemptydoublepage
\phantomsection
\section*{Bibliographie\footnote{
Version avec BibTeX}}
\addcontentsline{toc}{section}{Bibliographie}

\nocite{*} % optionnel : inclut toutes les refs du .bib même si non citées
\bibliographystyle{plain}
\bibliography{biblio} % fichier biblio.bib (même vide)

\end{document}